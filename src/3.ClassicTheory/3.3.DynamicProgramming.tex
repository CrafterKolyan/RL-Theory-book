\section{Динамическое программирование}\label{valueiterationsection}

\subsection{Метод простой итерации}

%\needspace{6\baselineskip}
\begin{wrapfigure}{r}{0.45\textwidth}
\vspace{-1cm}
\centering
\includegraphics[width=0.4\textwidth]{Images/ContractionMapping.png}
\vspace{-0.4cm}
\end{wrapfigure}

Мы увидели, что знание оценочных функций открывает путь к улучшению стратегии. Напрямую по определению считать их затруднительно; попробуем научиться решать уравнения Беллмана. И хотя уравнения оптимальности Беллмана нелинейные, они, тем не менее, имеют весьма определённый вид и, как мы сейчас увидим, обладают очень приятными свойствами. Нам понадобится несколько понятий внезапно из функана о том, как решать системы нелинейных уравнений вида $x \HM= f(x)$.

\begin{definition}
Оператор $f \colon X \to X$ называется \emph{сжимающим} (contraction) с коэффициентом сжатия $\gamma < 1$ по некоторой метрике $\rho$, если $\forall x_1, x_2 \colon$
$$\rho(f(x_1), f(x_2)) < \gamma \rho(x_1, x_2)$$
\end{definition}

\begin{definition}
Точка $x \in X$ для оператора $f \colon X \to X$ называется \emph{неподвижной} (fixed point), если
$$x = f(x)$$
\end{definition}

\begin{definition}
Построение последовательности $x_{k+1} = f(x_k)$ для начального приближения $x_0 \in X$ называется методом \emph{простой итерации} (point iteration) решения уравнения $x = f(x)$.
\end{definition}

\begin{theorem}[Теорема Банаха о неподвижной точке]\label{Banach}
В полном\footnote{любая фундаментальная последовательность имеет предел} метрическом пространстве $X$ у сжимающего оператора $f \colon X \to X$ существует и притом ровно одна неподвижная точка $x^*$, причём метод простой итерации сходится к ней из любого начального приближения.
\begin{proof}[Сходимость метода простой итерации] Пусть $x_0$ --- произвольное, $x_{k+1} = f(x_k)$. Тогда для любого $k > 0$:
\begin{equation}
\begin{aligned}\label{onestepcontraction}
\rho(x_k, x_{k+1}) = \{ \text{определение $x_k$} \} &= \rho(f(x_{k-1}), f(x_k)) \le \\ 
\le \{ \text{свойство сжатия} \} &\le \gamma \rho(x_{k-1}, x_k) \le \dots \le \\
\le \{ \text{аналогичным образом} \} &\le \dots \le \gamma^k \rho(x_0, x_1)
\end{aligned}
\end{equation}

Теперь посмотрим, что произойдёт после применения оператора $f$ $n$ раз:
\begin{align*}
\rho(x_k, x_{k + n}) &\le \\
\{ \text{неравенство треугольника} \}
&\le \rho(x_k, x_{k + 1}) + \rho(x_{k+1}, x_{k+2}) + \dots + \rho(x_{k + n - 1}, x_{k + n}) \le \\
\le \{ \text{\eqref{onestepcontraction}} \} 
&\le (\gamma^k + \gamma^{k+1} + \dots \gamma^{k+n-1}) \rho(x_0, x_1) \le \\
\le \{ \text{геом. прогрессия} \} &\le \frac{\gamma^k}{1 - \gamma} \rho(x_0, x_1) \xrightarrow{k \to \infty} 0
\end{align*}

Итак, последовательность $x_k$ --- фундаментальная, и мы специально попросили такое метрическое пространство (<<полное>>), в котором обязательно найдётся предел $x^* \coloneqq \lim\limits_{k \to \infty} x_k$.
\end{proof}
\begin{proof}[Существование неподвижной точки] Покажем, что $x^*$ и есть неподвижная точка $f$, то есть покажем, что наш метод простой итерации конструктивно её построил. Заметим, что для любого $k > 0$:
\begin{align*}
\rho(x^*, f(x^*)) &\le \\
\le \{ \text{неравенство треугольника} \} &\le \rho(x^*, x_k) + \rho(x_k, f(x^*)) = \\
= \{ \text{определение $x_k$} \} &= \rho(x^*, x_k) + \rho(f(x_{k-1}), f(x^*)) \le \\ 
\le \{ \text{свойство сжатия} \} &\le \rho(x^*, x_k) + \gamma \rho(x_{k-1}, x^*)
\end{align*}
Устремим $k \to \infty$; слева стоит константа, не зависящая от $k$. Тогда расстояние между $x_k$ и $x^*$ устремится к нулю, ровно как и между $x_{k-1}, x^*$ поскольку $x^*$ --- предел $x_k$. Значит, константа равна нулю, $\rho(x^*, f(x^*)) = 0$, следовательно, $x^* = f(x^*)$.
\end{proof}

\begin{proof}[Единственность] Пусть $x_1, x_2$ --- две неподвижные точки оператора $f$. Ну тогда:
$$\rho(x_1, x_2) = \rho(f(x_1), f(x_2)) \le \gamma \rho(x_1, x_2)$$
Получаем, что такое возможно только при $\rho(x_1, x_2) = 0$, то есть только если $x_1$ и $x_2$ совпадают.
\end{proof}
\end{theorem}

% TODO: пример, иллюстрация

\subsection{Policy Evaluation}

Вернёмся к RL. Известно, что $V^\pi$ для данного MDP и фиксированной политики $\pi$ удовлетворяет уравнению Беллмана \eqref{VV}. Для нас это система уравнений относительно значений $V^\pi(s)$. $V^\pi(s)$ --- объект (точка) в функциональном пространстве $\St \to \R$.

Будем решать её методом простой итерации\footnote{вообще говоря, это система линейных уравнений относительно значений $V^\pi(s)$, которую в случае табличных MDP можно решать любым методом решения СЛАУ. Однако, дальнейшие рассуждения через метод простой итерации обобщаются, например, на случай непрерывных пространств состояний $\St \HM\subseteq \R^n$.}. Для этого определим оператор $\B$, то есть преобразование из одной функции $\St \HM\to \R$ в другую. На вход этот оператор принимает функцию $V \colon \St \HM\subseteq \R^n$ и выдаёт некоторую другую функцию от состояний $\B V$. Чтобы задать выход оператора, нужно задать значение выходной функции в каждом $s \HM\in \St$; это значение мы будем обозначать $\left[\B V\right] (s)$ (квадратные скобки позволяют не путать применение оператора с вызовом самой функции) и определим его как правую часть решаемого уравнения \eqref{VV}. Итак:

\begin{definition}
Введём \emph{оператор Беллмана} (Bellman operator) для заданного MDP и стратегии $\pi$ как
\begin{equation}\label{bellmanoperatorV}
\left[\B V\right] (s) \coloneqq \E_{a \sim \pi(a \mid s)} \left[ r(s, a) + \gamma \E_{s'} V(s') \right]
\end{equation}
\end{definition}

Также нам нужна метрика на множестве функций $\St \to \R$; возьмём
$$d_\infty(V_1, V_2) \coloneqq \max_s | V_1(s) - V_2(s) |$$

\begin{theorem}
Если $\gamma < 1$, оператор $\B$ --- сжимающий с коэффициентом сжатия $\gamma$.
\beginproof
\begin{align*}
&d_\infty(\B V_1, \B V_2) = \max_s \left| [\B V_1](s) - [\B V_2](s) \right| = \\
&= \{ \text{подставляем значение операторов, т.е. правые части решаемого уравнения} \} = \\
&= \max_s \left| \E_{a} \left[ r(s, a) + \gamma \E_{s'} V_1(s') \right] - \E_{a} \left[ r(s, a) + \gamma \E_{s'} V_2(s') \right] \right| = \\
&= \{ \text{слагаемые $r(s, a)$ сокращаются} \} = \\
&= \gamma \max_s \left| \E_{a} \E_{s'} \left[ V_1(s') - V_2(s') \right] \right| \le \\
&\le \{ \text{используем свойство $\E_x f(x) \le \max\limits_x f(x)$} \} \le \\
&\le \gamma \max_s \max_{s'} \left| V_1(s') - V_2(s') \right| = \gamma d_\infty( V_1, V_2 ) \tagqed
\end{align*}
\end{theorem}

Итак, мы попали в теорему Банаха, и значит, метод простой итерации
$$V_{k+1} \coloneqq \B V_k$$
гарантированно сойдётся к единственной неподвижной точке при любой стартовой инициализации $V_0$. По построению мы знаем, что $V^\pi(s)$ такова, что $\B V^\pi = V^\pi$ (это и есть уравнение Беллмана), поэтому к ней и придём.

Важно помнить, что на каждой итерации такой процедуры текущее приближение не совпадает с истинной оценочной функцией: $V_k(s) \HM\approx V^\pi(s)$, но точного равенства поставить нельзя. Распространено (но, к сожалению, не везде применяется) соглашение обозначать аппроксимации оценочных функций без верхнего индекса: просто $V$ или $Q$. Однако, иногда, чтобы подчеркнуть, что алгоритм учит именно $V^\pi$, верхний индекс оставляют, что может приводить к путанице.

Обсудим, что случится в ситуации, когда $\gamma = 1$; напомним, что в таких ситуациях мы требовали эпизодичность сред, с гарантиями завершения всех эпизодов за $T^{\max}$ шагов. Оператор Беллмана формально сжатием являться уже не будет, и мы не подпадаем под теорему, поэтому этот случай придётся разобрать отдельно.

\begin{theoremBox}[label=th:policyevalepisodic]{}
В эпизодичных средах метод простой итерации сойдётся к единственному решению уравнений Беллмана не более чем за $T^{\max}$ шагов даже при $\gamma = 1$.
\begin{proof}
Мы уже доказывали теорему \ref{th:episodicmdpistree}, что граф таких сред является деревом. Будем говорить, что состояние $s$ находится на ярусе $T$, если при старте из $s$ у любой стратегии есть гарантии завершения за $T$ шагов. Понятно, что для состояния $s$ на ярусе $T$ верно, что $\forall s', a$, для которых $p(s' \mid s, a) > 0$, ярус $s'$ не превосходит $T - 1$. 

Осталось увидеть, что на $k$-ой итерации метода простой итерации вычисляет точные значения $V_k(s) \HM= V^{\pi}(s)$ для всех состояний на ярусах до $k$: действительно, покажем по индукции. Считаем, что терминальные состояния имеют нулевой ярус; а на $k$-ом шаге при обновлении $V_{k+1}(s) \HM\coloneqq \left[\B V_k\right](s)$ для $s$ на $k$-ом ярусе в правой части уравнения Беллмана будет стоять мат.~ожидание по $s'$ с ярусов до $k - 1$-го, для которых значение по предположению индукции уже посчитано точно.

Соответственно, за $T^{\max}$ шагов точные значения распространятся на все состояния, и конструктивно значения определены однозначно.
\end{proof}
\end{theoremBox}

\begin{remark}
Если $\gamma = 1$, а среда неэпизодична (такие MDP мы не допускали к рассмотрению), метод простой итерации может не сойтись, а уравнения Беллмана могут в том числе иметь бесконечно много решений. Пример подобного безобразия. Пусть в MDP без терминальных состояний с нулевой функцией награды (где, очевидно, $V^{\pi}(s) \HM= 0$ для всех $\pi, s$) мы проинициализировали $V_0(s) \HM= 100$ во всех состояниях $s$. Тогда при обновлении наша аппроксимация не будет меняться: мы уже в неподвижной точке уравнений Беллмана. В частности поэтому на практике практически никогда не имеет смысл выставлять $\gamma \HM = 1$, особенно в сложных средах, где, может быть, даже и есть эпизодичность, но, тем не менее, есть <<похожие состояния>>: они начнут работать <<как петли>>, когда мы перейдём к приближённым методам динамического программирования в дальнейшем.
\end{remark}

\begin{proposition}
Если некоторая функция $\tilde{V} \colon \St \to \R$ удовлетворяет уравнению Беллмана \eqref{VV}, то $\tilde{V} \equiv V^\pi$.
\end{proposition}

Мы научились решать задачу \emph{оценивания стратегии} (Policy Evaluation): вычислять значения оценочной функции по данной стратегии $\pi$ в ситуации, когда мы знаем динамику среды. На практике мы можем воспользоваться этим результатом только в \emph{<<табличном>> случае} (tabular RL), когда пространство состояний и пространство действий конечны и достаточно малы, чтобы все пары состояние-действие было возможно хранить в памяти компьютера и перебирать за разумное время. В такой ситуации $V^\pi(s)$ --- конечный векторочек, и мы умеем считать оператор Беллмана и делать обновления $V_{k+1} \HM= \B V_k$.

\begin{algorithm}[label=policyevaluation]{Policy Evaluation}
\textbf{Вход:} $\pi(a \mid s)$ --- стратегия \\
\textbf{Гиперпараметры:} $\eps$ --- критерий останова

\vspace{0.3cm}
Инициализируем $V_0(s)$ произвольно для всех $s \in \St$ \\
\textbf{На $k$-ом шаге:}
\begin{enumerate}
    \item $\forall s \colon V_{k+1}(s) \coloneqq \E_{a} \left[ r(s, a) + \gamma \E_{s'} V_k(s')\right]$
    \item \textbf{критерий останова:} $\max\limits_s |V_k(s) - V_{k+1}(s)| < \eps$
\end{enumerate}

\vspace{0.3cm}
\textbf{Выход:} $V_k(s)$
\end{algorithm}

\begin{exampleBox}[label=ex:pe, righthand ratio=0.25, sidebyside, sidebyside align=center, lower separated=false]{}
Проведём оценивание стратегии, случайно выбирающей, в какую сторону ей пойти, с $\gamma = 0.9$. Угловые клетки с ненулевой наградой терминальны; агент остаётся в той же клетке, если упирается в стенку. На каждой итерации отображается значение текущего приближения $V_k(s) \HM\approx V^{\pi}(s)$.

\tcblower
%\vspace{-0.3cm}
\animategraphics[controls, width=0.9\linewidth]{1}{Images/PE/policyeval}{0}{9}
\end{exampleBox}

Итак, мы научились считать $V^\pi$ в предположении известной динамики среды. Полностью аналогичное рассуждение верно и для уравнений QQ \eqref{QQ}; то есть, расширив набор переменных, в табличных MDP можно методом простой итерации находить $Q^{\pi}$ и <<напрямую>>. Пока модель динамики среды считается известной, это не принципиально: мы можем посчитать и Q-функцию через V-функцию по формуле QV \eqref{QV}.


% Удобен схожий алгоритм\footnote{не могу найти внятного объяснения, почему этот переход ничего не ломает, но кажется, что это так.}, позволяющий не хранить целиком $V_k$ для расчёта $V_{k+1}$, а проводить все вычисления по одной таблице:

% \begin{algorithm}{Iterative Policy Evaluation}
% \textbf{Вход:} $\pi(a \mid s)$ --- стратегия \\
% \textbf{Гиперпараметры: } $\varepsilon$ --- критерий останова

% \vspace{0.3cm}
% Инициализируем $V(s)$ произвольно для всех $s \in \St$ \\
% \textbf{На $k$-ом шаге:}
% \begin{enumerate}
%     \item храним максимальное изменение на данном шаге $\delta_k \coloneqq 0$
%     \item \textbf{перебираем $s \in \St$:}
%     \begin{itemize}
%         \item сохраняем старое значение $\nu \coloneqq V(s)$
%         \item обновляем $V(s) \leftarrow \E_{a} \left[ r(s, a) + \gamma \E_{s'} V(s')\right]$
%         \item обновляем максимальное изменение $\delta_k \leftarrow \max \left( \delta_k, |V(s) - \nu| \right)$
%     \end{itemize}
%     \item \textbf{критерий останова:} $\delta_k < \varepsilon$
% \end{enumerate}

% \vspace{0.3cm}
% \textbf{Выход:} $V(s)$
% \end{algorithm}

\subsection{Value Iteration}

Теорема Банаха позволяет аналогично Policy Evaluation (алг. \ref{policyevaluation}) решать уравнения оптимальности Беллмана \eqref{Q*Q*} через метод простой итерации. Действительно, проведём аналогичные рассуждения (мы сделаем это для $Q^*$, но совершенно аналогично можно было бы сделать это и для $V^*$):

\begin{definition}
Определим \emph{оператор оптимальности Беллмана} (Bellman optimality operator, Bellman control operator) $\B^*$:
$$\left[\B^* Q\right] (s, a) \coloneqq r(s, a) + \gamma \E_{s'} \max_{a'}Q(s', a')$$
\end{definition}

В качестве метрики на множестве функций $\St \times \A \to \R$ аналогично возьмём
$$d_\infty(Q_1, Q_2) \coloneqq \max_{s, a} | Q_1(s, a) - Q_2(s, a) |$$

Нам понадобится следующий факт:
\begin{proposition}\,
\begin{equation}\label{diffmax}
| \max_x f(x) - \max_x g(x) | \le \max_x | f(x) - g(x) |
\end{equation}
\begin{proof}
Рассмотрим случай $\max\limits_x f(x) \HM> \max\limits_x g(x)$. Пусть $x^*$ --- точка максимума $f(x)$. Тогда:
\begin{align*}
    \max_x f(x) - \max_x g(x) \le f(x^*) - \max_x g(x) \le f(x^*) - g(x^*) \le \max_x | f(x) - g(x) |
\end{align*}
Второй случай рассматривается симметрично.
\end{proof}
\end{proposition}

\begin{theorem}
Если $\gamma < 1$, оператор $\B^*$ --- сжимающий.
\beginproof
\begin{align*}
&d_\infty(\B^*Q_1, \B^*Q_2) = \max_{s, a} \left| [\B^*Q_1](s, a) - [\B^*Q_2](s, a) \right| = \\
&= \{ \text{подставляем значения операторов, т.е. правые части решаемой системы уравнений} \} = \\
&= \max_{s, a} \left| \left[ r(s, a) + \gamma \E_{s'} \max_{a'} Q_1(s', a') \right] - \left[ r(s, a) + \gamma \E_{s'} \max_{a'} Q_2(s', a') \right] \right| = \\
&= \{ \text{слагаемые $r(s, a)$ сокращаются} \} = \\
&= \gamma \max_{s, a} \left| \E_{s'} \left[ \max_{a'} Q_1(s', a') - \max_{a'} Q_2(s', a') \right] \right| \le \\
&\le \{ \text{используем свойство $\E_x f(x) \le \max\limits_x f(x)$} \} \le \\
&\le \gamma \max_{s, a} \max_{s'} \left| \max_{a'} Q_1(s', a') - \max_{a'} Q_2(s', a') \right| = \\
&\le \{ \text{используем свойство максимумов \eqref{diffmax}} \} \le \\
& \le \gamma \max_{s, a} \max_{s'} \max_{a'} \left| Q_1(s', a') - Q_2(s', a') \right| \le \\
&= \{\text{внутри стоит определение $d_\infty(Q_1, Q_2)$, а от внешнего максимума ничего не зависит}\} = \\
&= \gamma d_\infty( Q_1, Q_2 )   \tagqed
\end{align*}
\end{theorem}

\begin{theorem}
В эпизодичных средах метод простой итерации сойдётся к единственному решению уравнений оптимальности Беллмана не более чем за $T^{\max}$ шагов даже при $\gamma = 1$.
\begin{proof} Полностью аналогично доказательству теоремы \ref{th:policyevalepisodic}.
\end{proof}
\end{theorem}

\begin{proposition}
Если некоторая функция $\tilde{Q} \colon \St \times \A \to \R$ удовлетворяет уравнению оптимальности Беллмана \eqref{Q*Q*}, то $\tilde{Q} \equiv Q^*$.
\end{proposition}

\begin{proposition}
Метод простой итерации сходится к $Q^*$ из любого начального приближения.
\end{proposition}

Вообще, если известна динамика среды, то нам достаточно решить уравнения оптимальности для $V^*$ --- это потребует меньше переменных. Итак, в табличном случае мы можем напрямую методом простой итерации решать уравнения оптимальности Беллмана и в пределе сойдёмся к оптимальной оценочной функции, которая тут же даёт нам оптимальную стратегию.

\begin{algorithm}[label=valueiteration]{Value Iteration}
\textbf{Вход:} $\varepsilon$ --- критерий останова

\vspace{0.3cm}
Инициализируем $V_0(s)$ произвольно для всех $s \in \St$ \\
\textbf{На $k$-ом шаге:}
\begin{enumerate}
    \item для всех $s$: $V_{k+1}(s) \coloneqq \max\limits_a \left[ r(s, a) + \gamma \E_{s'} V_k(s')\right]$
    \item \textbf{критерий останова:} $\max\limits_{s} | V_{k+1}(s) - V_k(s) | < \varepsilon$
\end{enumerate}

\vspace{0.3cm}
\textbf{Выход:} $\pi(s) \coloneqq \argmax\limits_a \left[ r(s, a) + \gamma \E_{s'} V(s')\right]$
\end{algorithm}

%\needspace{7\baselineskip}
\begin{wrapfigure}{r}{0.35\textwidth}
%\vspace{-0.3cm}
\centering
\includegraphics[width=0.3\textwidth]{Images/DP_backup.png}
%\vspace{-0.3cm}
\end{wrapfigure}

Итак, мы придумали наш первый табличный алгоритм планирования --- алгоритм, решающий задачу RL в условиях известной модели среды. На каждом шаге мы обновляем (<<бэкапим>>) нашу текущую аппроксимацию V-функции на её \emph{одношаговое приближение} (one-step approximation): смотрим на один шаг в будущее ($a, r, s'$) и приближаем всё остальное будущее текущей же аппроксимацией. Такой <<бэкап динамического программирования>> (dynamic programming backup, DP-backup) --- обновление <<бесконечной ширины>>: мы должны перебрать все возможные варианты следующего одного шага, рассмотреть все свои действия (по ним мы возьмём максимум) и перебрать всевозможные ответы среды --- $s'$ (по ним мы должны рассчитать мат.ожидание). Поэтому этот алгоритм в чистом виде напоминает то, что обычно и понимается под словами <<динамическое программирование>>: мы <<раскрываем дерево игры>> полностью на один шаг вперёд.

\begin{exampleBox}[righthand ratio=0.25, sidebyside, sidebyside align=center, lower separated=false]{}
Решим задачу из примера \ref{ex:pe}, $\gamma = 0.9$; на каждой итерации отображается значение текущего приближения $V_k(s) \HM\approx V^{*}(s)$. В конце концов в силу детермнированности среды станет понятно, что можно избежать попадания в терминальное -1 и кратчайшим путём добираться до терминального +1.

\tcblower
%\vspace{-0.3cm}
\animategraphics[controls, width=0.9\linewidth]{1}{Images/VI/valueiter}{0}{9}
\end{exampleBox}

% \begin{algorithm}[label=valueiteration]{Value Iteration}
% \textbf{Вход:} $\varepsilon$ --- критерий останова

% \vspace{0.3cm}
% Инициализируем $V^*(s)$ произвольно для всех $s \in \St$ \\
% \textbf{На $k$-ом шаге:}
% \begin{enumerate}
%     \item храним максимальное изменение $\delta_k \coloneqq 0$
%     \item \textbf{перебираем $s \in \St$:}
%     \begin{itemize}
%         \item сохраняем старое значение $\nu \coloneqq V^*(s)$
%         \item обновляем $V^*(s) \leftarrow \max\limits_a \left[ r(s, a) + \gamma \E_{s'} V^*(s')\right]$
%         \item обновляем максимальное изменение $\delta_k \leftarrow \max \left( \delta_k, |V^*(s) - \nu| \right)$
%     \end{itemize}
%     \item \textbf{критерий останова:} $\delta_k < \varepsilon$
% \end{enumerate}

% \vspace{0.3cm}
% \textbf{Выход:} $\pi(s) \coloneqq \argmax\limits_a \left[ r(s, a) + \gamma \E_{s'} V^*(s')\right]$
% \end{algorithm}

\subsection{Policy Iteration}

\needspace{7\baselineskip}
\begin{wrapfigure}{r}{0.35\textwidth}
\vspace{-1.3cm}
\centering
\includegraphics[width=0.35\textwidth]{Images/PI_basic.png}
\vspace{-1.3cm}
\end{wrapfigure}

Мы сейчас в некотором смысле <<обобщим>> Value Iteration и придумаем более общую схему алгоритма планирования для табличного случая.

Для очередной стратегии $\pi_k$ посчитаем её оценочную функцию $Q^{\pi_k}$, а затем воспользуемся теоремой Policy Improvement \ref{th:policyimprovement} и построим стратегию лучше; например, жадно:
$$\pi_{k+1}(s) \coloneqq \argmax\limits_{a} Q^{\pi_k}(s, a)$$

Тогда у нас есть второй алгоритм планирования, который, причём, перебирает детерминированные стратегии, обладающие свойством монотонного возрастания качества: каждая следующая стратегия не хуже предыдущей. Он работает сразу в классе детерминированных стратегий, и состоит из двух этапов:
\begin{itemize}
    \item \textbf{Policy Evaluation}: вычисление $Q^\pi$ для текущей стратегии $\pi$;
    \item \textbf{Policy Improvement}: улучшение стратегии $\pi(s) \leftarrow \argmax\limits_a Q^\pi(s, a)$;
\end{itemize}

При этом у нас есть гарантии, что когда алгоритм <<останавливается>> (не может провести Policy Improvement), то он находит оптимальную стратегию. Будем считать\footnote{считаем, что аргмакс берётся однозначно для любой Q-функции: в случае, если в $\Argmax$ содержится более одного элемента, множество действий как-то фиксированно упорядочено, и берётся действие с наибольшим приоритетом.}, что в такой момент остановки после проведения Policy Improvement наша стратегия не меняется: $\pi_{k+1} \equiv \pi_k$.

\begin{theorem}
В табличном сеттинге Policy Iteration завершает работу за конечное число итераций.
\begin{proof}
Алгоритм перебирает детерминированные стратегии, и, если остановка не происходит, каждая следующая лучше предыдущей:
$$\pi_k \succ \pi_{k-1} \succ \dots \succ \pi_0$$
Это означает, что все стратегии в этом ряду различны. Поскольку в табличном сеттинге число состояний и число действий конечны, детерминированных стратегий конечное число; значит, процесс должен закончится.
\end{proof}
\end{theorem}

\begin{algorithm}[label=policyiteration]{Policy Iteration}
\textbf{Гиперпараметры:} $\eps$ --- критерий останова для процедуры $\operatorname{PolicyEvaluation}$

\vspace{0.3cm}
Инициализируем $\pi_0(s)$ произвольно для всех $s \in \St$ \\
\textbf{На $k$-ом шаге:}
\begin{enumerate}
    \item $V^{\pi_k} \coloneqq \operatorname{PolicyEvaluation}(\pi_k, \eps)$
    \item $Q^{\pi_k}(s, a) \coloneqq r(s, a) + \gamma \E_{s'} V^{\pi_k}(s')$
    \item $\pi_{k+1}(s) \coloneqq \argmax\limits_a Q^{\pi_k}(s, a)$
    \item \textbf{критерий останова:} $\pi_k \equiv \pi_{k+1}$
\end{enumerate}
\end{algorithm}

\begin{exampleBox}[righthand ratio=0.5, sidebyside, sidebyside align=center, lower separated=false]{}
Решим задачу из примера \ref{ex:pe}, $\gamma = 0.9$; на каждой итерации слева отображается $V^{\pi_k}(s)$; справа улучшенная $\pi_{k+1}$. За 4 шага алгоритм сходится к оптимальной стратегии.

\tcblower
%\vspace{-0.3cm}
\animategraphics[controls, width=\linewidth]{1}{Images/PI/policyiter}{0}{3}
\end{exampleBox}

\subsection{Generalized Policy Iteration}

\needspace{9\baselineskip}
\begin{wrapfigure}{r}{0.35\textwidth}
\vspace{-1.3cm}
\centering
\includegraphics[width=0.3\textwidth]{Images/VI_basic.png}
\vspace{-0.7cm}
\end{wrapfigure}

Policy Iteration --- идеализированный алгоритм: на этапе оценивания в табличном сеттинге можно попробовать решить систему уравнений Беллмана $V^\pi$ с достаточно высокой точностью за счёт линейности этой системы уравнений, или можно считать, что проводится достаточно большое количество итераций метода простой итерации. Тогда, вообще говоря, процедура предполагает бесконечное число шагов, и на практике нам нужно когда-то остановиться; теоретически мы считаем, что доводим вычисления до некоторого критерия останова, когда значения вектора не меняются более чем на некоторую погрешность $\eps \HM> 0$. 

Но рассмотрим такую, пока что, эвристику: давайте останавливать Policy Evaluation после ровно $N$ шагов, а после обновления стратегии не начинать оценивать $\pi_{k+1}$ с нуля, а использовать последнее $V(s) \HM\approx V^{\pi_k}$ в качестве инициализации. Тогда наш алгоритм примет следующий вид: 

\begin{algorithm}[label=generalizedpolicyiteration]{Generalized Policy Iteration}
\textbf{Гиперпараметры:} $N$ --- количество шагов

\vspace{0.3cm}
Инициализируем $\pi(s)$ произвольно для всех $s \in \St$ \\
Инициализируем $V(s)$ произвольно для всех $s \in \St$ \\ 
\textbf{На $k$-ом шаге:}
\begin{enumerate}
    \item \textbf{Повторить $N$ раз:}
    \begin{itemize}
        \item $\forall s \colon V(s) \leftarrow \E_{a} \left[ r(s, a) + \gamma \E_{s'} V(s')\right]$
    \end{itemize}
    \item $Q(s, a) \leftarrow r(s, a) + \gamma \E_{s'} V(s')$
    \item $\pi(s) \leftarrow \argmax\limits_a Q(s, a)$
\end{enumerate}
\end{algorithm}

Мы формально теряем гарантии улучшения стратегии на этапе Policy Improvement, поэтому останавливать алгоритм после того, как стратегия не изменилась, уже нельзя: возможно, после следующих $N$ шагов обновления оценочной функции, аргмакс поменяется, и стратегия всё-таки сменится. Но такая схема в некотором смысле является наиболее общей, и вот почему:

\begin{proposition}
Generalized Policy Iteration (алг. \ref{generalizedpolicyiteration}) совпадает с Value Iteration (алг. \ref{valueiteration}) при $N \HM= 1$ и с Policy Iteration (алг. \ref{policyiteration}) при $N = \infty$.
\begin{proof}
Второе очевидно; увидим первое. При $N = 1$ наше обновление V-функции имеет следующий вид:
$$V(s) \leftarrow \E_{a} \left[ r(s, a) + \gamma \E_{s'} V(s') \right]$$
Вспомним, по какому распределению берётся мат.~ожидание $\E_a$: по $\pi$, которая имеет вид 
$$\pi(s) = \argmax\limits_a Q(s, a) = \argmax\limits_a \left[ r(s, a) + \gamma \E_{s'} V(s') \right]$$
Внутри аргмакса как раз стоит содержимое нашего мат.ожидания в обновлении V, поэтому это обновление выродится в
$$V(s) \leftarrow \max\limits_{a} \left[ r(s, a) + \gamma \E_{s'} V(s') \right]$$
Это в точности обновление из алгоритма Value Iteration.
\end{proof}
\end{proposition}

Итак, Generalized Policy Iteration при $N \HM= 1$ и при $N \HM= \infty$ --- это ранее разобранные алгоритмы, физический смысл которых нам ясен. В частности, теперь понятно, что в Value Iteration очередное приближение $Q_k(s, a) \HM\approx Q^*(s,a)$ можно также рассматривать как приближение $Q_k(s, a) \HM \approx Q^{\pi}(s, a)$ для $\pi(s) \HM \coloneqq \argmax\limits_{a} Q_k(s, a)$; то есть в алгоритме хоть и не потребовалось в явном виде хранить <<текущую>> стратегию, она всё равно неявно в нём присутствует. 

Давайте попробуем понять, что происходит в Generalized Policy Iteration при промежуточных $N$. Заметим, что повторение $N$ раз шага метода простой итерации для решения уравнения $\B V^\pi \HM= V^\pi$ эквивалентно одной итерации метода простой итерации для решения уравнения $\B^N V^\pi \HM= V^\pi$ (где запись $\B^N$ означает повторное применение оператора $\B$ $N$ раз), для которого, очевидно, искомая $V^{\pi}$ также будет неподвижной точкой. Что это за оператор $\B^N$? 

В уравнениях Беллмана мы <<раскручивали>> наше будущее на один шаг вперёд и дальше заменяли оставшийся <<хвост>> на определение V-функции. Понятно, что мы могли бы раскрутить не на один шаг, а на $N$ шагов вперёд.

\begin{theorem}[$N$-шаговое уравнение Беллмана]\,
\begin{equation}\label{NstepBellman}
 V^{\pi}(s_0) = \E_{\Traj_{:N} \sim \pi \mid s_0} \left[ \sum_{t=0}^{N-1} \gamma^{t}r_t +  \gamma^N \E_{s_N} V^{\pi}(s_N) \right]   
\end{equation}
\begin{proof}[Доказательство по индукции]
Для получения уравнения на $N$ шагов берём $N-1$-шаговое и подставляем в правую часть раскрутку на один шаг из уравнения \eqref{VV}. Это в точности соответствует применению оператора Беллмана $N$ раз.
\end{proof}
\begin{proof}[Доказательство без индукции]
Для любых траекторий $\Traj$ верно, что
$$R(\Traj) = \sum_{t=0}^{N-1} \gamma^{t}r_t + \gamma^N R_N$$
Возьмём мат.ожидание $\E_{\Traj \sim \pi \mid s_0}$ слева и справа:
$$\E_{\Traj \sim \pi \mid s_0}R(\Traj) = \E_{\Traj \sim \pi \mid s_0} \left[ \sum_{t=0}^{N-1} \gamma^{t}r_t + \gamma^N R_N \right]$$
Слева видно определение V-функции. Справа достаточно разделить мат.ожидание на мат.ожидание по первым $N$ шагам и хвост:
$$V^{\pi}(s_0) = \E_{\Traj_{:N} \sim \pi \mid s_0} \left[ \sum_{t=0}^{N-1} \gamma^{t}r_t + \gamma^N \E_{s_N} \E_{\Traj_{N:} \sim \pi \mid s_N} R_N \right]$$
Осталось выделить справа во втором слагаемом определение V-функции.
\end{proof}
\end{theorem}

\begin{proposition}
$\B^N$ --- оператор с коэффициентом сжатия $\gamma^N$.
\beginproof
\begin{align*}
\rho(\B^N V_1, \B^N V_2) \le \gamma \rho (\B^{N-1} V_1, \B^{N-1} V_2) \le \dots \le \gamma^N \rho (V_1, V_2)   \tagqed
\end{align*}
\end{proposition}

Означает ли это, что метод простой итерации решения $N$-шаговых уравнений сойдётся быстрее? Мы по сути просто <<за один шаг>> делаем $N$ итераций метода простой итерации для решения обычного одношагового уравнения; в этом смысле, мы ничего не выигрываем. В частности, если мы устремим $N$ к бесконечности, то мы получим просто определение V-функции; формально, в правой части будет стоять выражение, вообще не зависящее от поданной на вход оператору $V(s)$, коэффициент сжатия будет ноль, и метод простой итерации как бы сходится тут же за один шаг. Но для проведения этого шага нужно выинтегрировать все траектории --- <<раскрыть дерево полностью>>.

Но теперь у нас есть другой взгляд на Generalized Policy Iteration: мы чередуем одну итерацию решения $N$-шагового уравнения Беллмана с Policy Improvement-ом.

\begin{theorem}
Алгоритм Generalized Policy Iteration \ref{generalizedpolicyiteration} при любом $N$ сходится к оптимальной стратегии и оптимальной оценочной функции.
\begin{proof}[Без доказательства]
\end{proof}
\end{theorem}

\needspace{7\baselineskip}
\begin{wrapfigure}{r}{0.35\textwidth}
\vspace{-0.7cm}
\centering
\includegraphics[width=0.3\textwidth]{Images/GPI.png}
\vspace{-0.5cm}
\end{wrapfigure}

Интуитивно, такой алгоритм <<стабилизируется>>, если оценочная функция будет удовлетворять уравнению Беллмана для текущей $\pi$ (иначе оператор $\B^N$ изменит значение функции), и если $\pi$ выбирает $\argmax\limits_{a} Q(s, a)$ из неё; а если аппроксимация V-функции удовлетворяет уравнению Беллмана, то она совпадает с $V^\pi$, и значит $Q(s, a) \HM= Q^\pi(s, a)$. То есть, при сходимости стратегия $\pi$ будет выбирать действие жадно по отношению к своей же $Q^\pi$, а мы помним, что это в точности критерий оптимальности.

Все алгоритмы, которые мы будем обсуждать далее, так или иначе подпадают под обобщённую парадигму <<оценивание-улучшение>>. У нас будет два процесса оптимизации: обучение \emph{актёра} (actor), политики $\pi$, и \emph{критика} (critic), оценочной функции (Q или V). Критик обучается оценивать текущую стратегию, текущего актёра: сдвигаться в сторону решения какого-нибудь уравнения, для которого единственной неподвижной точкой является $V^{\pi}$ или $Q^{\pi}$. Актёр же будет учиться при помощи policy improvement-а: вовсе не обязательно делать это жадно, возможно учиться выбирать те действия, где оценка критика <<побольше>>, оптимизируя в каких-то состояниях (в каких - пока открытый вопрос) функционал \eqref{pi_optimization}:
$$\E_{\pi(a \mid s)} Q(s, a) \to \max_{\pi}$$

Причём, возможно, в этом функционале нам не понадобится аппроксимация (модель) Q-функции в явном виде, и тогда мы можем обойтись лишь какими-то оценками $Q^{\pi}$; в таким ситуациях нам достаточно будет на этапе оценивания политики обучать лишь модель $V^{\pi}$ для текущей стратегии. А, например, в эволюционных методах мы обошлись вообще без обучения критика именно потому, что смогли обойтись лишь Монте-Карло оценками будущих наград. Этот самый простой способ решать задачу RL --- погенерировать несколько случайных стратегий и выбрать среди них лучшую --- тоже условно подпадает под эту парадигму: мы считаем Монте-Карло оценки значения $J(\pi)$ для нескольких разных стратегий (evaluation) и выбираем наилучшую стратегию (improvement). Поэтому Policy Improvement, как мы увидим, тоже может выступать в разных формах: например, возможно, как в Value Iteration, у нас будет приближение Q-функции, и мы будем просто всегда полагать, что policy improvement проводится жадно, и текущей стратегией неявно будет $\argmax\limits_a Q(s, a)$.

Но главное, что эти два процесса, оценивание политики (обучение критика) и улучшение (обучение актёра) можно будет проводить стохастической оптимизацией. Достаточно, чтобы лишь \textit{в среднем} модель оценочной функции сдвигалась в сторону $V^{\pi}$ или $Q^{\pi}$, а актёр лишь \textit{в среднем} двигался в сторону $\argmax\limits_a Q(s, a)$. И такой <<рецепт>> алгоритма всегда будет работать: пока оба этих процесса проводятся корректно, итоговый алгоритм запустится на практике. Это в целом фундаментальная идея всего RL. В зависимости от выбора того, как конкретно проводить эти процессы, получатся разные по свойствам алгоритмы, и, в частности, отдельно интересными будут алгоритмы, <<схлопывающие>> схему Generalized Policy Iteration в её предельную форму, в Value Iteration. 

Мы далее начнём строить model-free алгоритмы, взяв наши алгоритмы планирования --- Policy Iteration и Value Iteration, --- и попробовав превратить их в табличные алгоритмы решения задачи.

% Исключительно для удобства нотации, покажем сходимость такого алгоритма в случае, если мы учим сразу Q-функцию, то есть проводим обновления следующего вида:
% $$Q^{\pi}(s_0, a_0) \leftarrow \E_{\Traj_{:N} \sim \pi \mid s_0, a_0} \left[ \sum_{t=0}^{N-1} \gamma^{t}r(s_t, a_t) + \gamma^N \E_{s_N} \E_{a_N} Q^{\pi}(s_N, a_N) \right],$$
% где $\pi(s) \coloneqq \argmax\limits_{a} Q^{\pi}(s, a)$, и, следовательно, обновление принимает вид:
% $$Q^{\pi}(s_0, a_0) \leftarrow \E_{\Traj_{:N} \sim \pi \mid s_0, a_0} \left[ \sum_{t=0}^{N-1} \gamma^{t}r(s_t, a_t) + \gamma^N \E_{s_N} \max\limits_{a_N} Q^{\pi}(s_N, a_N) \right],$$

% Можно ли к правой части применить нашу технику со сжимающими операторами? Выражение в правой части зависит как от нашей текущей аппроксимации $Q^{\pi}$, так и от стратегии $\pi$. 

% \begin{theorem}
% Generalized Policy Iteration, заключающийся в чередовании обновлений
% \begin{gather*}
% Q^{\pi} \leftarrow \B^{\pi}_N Q^{\pi} \\
% \pi(s) \leftarrow \argmax_{a} Q^{\pi}(s, a),
% \end{gather*}
% где
% $$\left[ \B^{\pi}_N Q^{\pi} \right](s, a) \coloneqq \E_{\Traj_{:N} \sim \pi \mid s_0, a_0} \left[ \sum_{t=0}^{N-1} \gamma^{t}r(s_t, a_t) + \gamma^N \E_{s_N} \max\limits_{a_N} Q^{\pi}(s_N, a_N) \right],$$
% сходится к $Q^*$.
% \begin{proof}
% Тонкость заключается в том, что наш оператор изменяется на каждом шаге (всё время меняется $\pi$). Однако, каждый оператор $\B^{\pi}_N$ является сжимающим в $\gamma^N$ раз: действительно,
% $$\B^{\pi}_N \equiv \B^{N-1} \B^*,$$
% а композиция сжимающих операторов есть также сжатие.

% Что за неподвижная точка у оператора $\B^{\pi}_N$ для фиксированной $\pi$? $Q^\pi$!
% \end{proof}
% \end{theorem}